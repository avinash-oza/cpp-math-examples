\documentclass{article}
\usepackage{amsmath,nccmath}
\usepackage[landscape]{geometry}

\begin{document}
\section*{Problem 6}
% PART i
i) To find the asset allocation of the tangency portfolio, we use the minimum variance portfolio from tangency portfolio function (p256 from the book) with:
\begin{equation*}
\Sigma_{R}=% latex table generated in R 3.1.1 by xtable 1.7-4 package
% Mon Mar 23 18:30:37 2015
\begin{bmatrix}{}
 0.022500 & 0.007500 \\ 
  0.007500 & 0.040000 \\ 
  \end{bmatrix},
 \mu = \begin{bmatrix}{} 
0.08 \\
0.12 \\
  \end{bmatrix}
\end{equation*}
We obtain that:
\begin{equation*}
w_{T} = % latex table generated in R 3.1.1 by xtable 1.7-4 package
% Mon Mar 23 18:33:34 2015
\begin{bmatrix}{}
 0.333333 \\ 
  0.666667 \\ 
  \end{bmatrix}
\end{equation*}  
which means allocate \$3,333,333 to asset 1 and \$6,666,666 to asset 2.
\vspace{5mm} \\
% PART ii
%%
ii) To find the asset allocation for a minimum variance portfolio with 7\% expected return, we use Table 9.1 on p255 with:
\begin{equation*}
\Sigma_{R}=% latex table generated in R 3.1.1 by xtable 1.7-4 package
% Mon Mar 23 18:30:37 2015
\begin{bmatrix}{}
 0.022500 & 0.007500 \\ 
  0.007500 & 0.040000 \\ 
  \end{bmatrix},
 \mu = \begin{bmatrix}{} 
0.08 \\
0.12 \\
  \end{bmatrix}
\end{equation*}
We obtain that:
\begin{equation*}
w = % latex table generated in R 3.1.1 by xtable 1.7-4 package
% Mon Mar 23 18:36:49 2015
\begin{bmatrix}{}
 0.117647 \\ 
  0.235294 \\ 
  \end{bmatrix},
w_{cash} = .647059,
\sigma_{portfolio} = 0.0542326,
\end{equation*}
which means allocate long \$1,176,470 asset 1, long \$2,352,940 asset 2 and long \$6,470,590 cash. The standard deviation of the portfolio is approx 5.42\%.
\vspace{5mm} \\
% PART iii
%%%%
iii) To find the asset allocation for a minimum variance portfolio with 11\% expected return, we use Table 9.1 on p255 with:
\begin{equation*}
\Sigma_{R}=% latex table generated in R 3.1.1 by xtable 1.7-4 package
% Mon Mar 23 18:30:37 2015
\begin{bmatrix}{}
 0.022500 & 0.007500 \\ 
  0.007500 & 0.040000 \\ 
  \end{bmatrix},
 \mu = \begin{bmatrix}{} 
0.08 \\
0.12 \\
  \end{bmatrix}
\end{equation*}
We obtain that:
\begin{equation*}
w = % latex table generated in R 3.1.1 by xtable 1.7-4 package
% Mon Mar 23 18:46:12 2015
\begin{bmatrix}{}
 0.352941 \\ 
  0.705882 \\ 
  \end{bmatrix},
w_{cash} = -0.058824,
\sigma_{portfolio} = 0.162698
\end{equation*}
which means long \$3,529,410 asset 1, long \$7,058,820 asset 2 and short \$588,240 cash. The standard deviation of the portfolio is approx 16.2\%.
\vspace{5mm} \\
% PART iv
%%%%
iv) To find the asset allocation for a maximum return portfolio with 12\% standard deviation, we use Table 9.3 on p257 with:
\begin{equation*}
\Sigma_{R}=% latex table generated in R 3.1.1 by xtable 1.7-4 package
% Mon Mar 23 18:30:37 2015
\begin{bmatrix}{}
 0.022500 & 0.007500 \\ 
  0.007500 & 0.040000 \\ 
  \end{bmatrix},
 \mu = \begin{bmatrix}{} 
0.08 \\
0.12 \\
  \end{bmatrix}
\end{equation*}
We obtain that:
\begin{equation*}
w = % latex table generated in R 3.1.1 by xtable 1.7-4 package
% Mon Mar 23 18:56:24 2015
\begin{bmatrix}{}
 0.260317 \\ 
  0.520633 \\ 
  \end{bmatrix},
w_{cash} = 0.219050,
\mu_{portfolio} = 0.094254
\end{equation*}
which means long \$2,603,170 asset 1, long \$5,206,330 asset 2 and long \$2,190,500 cash. The expected return of the portfolio is approx 9.42\%.
\vspace{5mm} \\
% PART v
%%%%
v) To find the asset allocation for a maximum return portfolio with 18\% standard deviation, we use Table 9.3 on p257 with:
\begin{equation*}
\Sigma_{R}=% latex table generated in R 3.1.1 by xtable 1.7-4 package
% Mon Mar 23 18:30:37 2015
\begin{bmatrix}{}
 0.022500 & 0.007500 \\ 
  0.007500 & 0.040000 \\ 
  \end{bmatrix},
 \mu = \begin{bmatrix}{} 
0.08 \\
0.12 \\
  \end{bmatrix}
\end{equation*}
We obtain that:
\begin{equation*}
w = % latex table generated in R 3.1.1 by xtable 1.7-4 package
% Mon Mar 23 19:03:33 2015
\begin{bmatrix}{}
 0.390475 \\ 
  0.780950 \\ 
  \end{bmatrix},
w_{cash} = -0.171424,
\mu_{portfolio} = 0.116381
\end{equation*}
which means long \$3,904,750 asset 1, long \$7,809,500 asset 2 and short \$1,714,240 cash. The expected return of the portfolio is approx 11.63\%.
\vspace{5mm} \\
% PART vi
%%%%
vi) With the risk free rate changing to 5.25\%, and our portfolio required to have a 7\% expected return: 
\begin{equation*}
w_{new} = % latex table generated in R 3.1.1 by xtable 1.7-4 package
% Mon Mar 23 19:09:18 2015
\begin{bmatrix}{}
 0.099032 \\ 
  0.218913 \\ 
  \end{bmatrix},
w_{new,cash} = 0.682055,
\sigma_{portfolio} = 0.049626
\end{equation*}
The weights from part ii were:
\begin{equation*}
w = % latex table generated in R 3.1.1 by xtable 1.7-4 package
% Mon Mar 23 18:36:49 2015
\begin{bmatrix}{}
 0.117647 \\ 
  0.235294 \\ 
  \end{bmatrix},
w_{cash} = 0.647059,
\end{equation*}
which means we would decrease our allocation to asset 1 from \$1,176,470 to \$990,320, decrease our allocation to asset 2 from \$2,352,940 to \$2,189,130 asset 2 and increase our cash from \$6,470,590 to \$6,820,550.
%\begin{equation*}
%\end{equation*}

\end{document}
