\documentclass{article}
\usepackage{amsmath,nccmath}

\usepackage[landscape]{geometry}

\begin{document}
\section*{Problem 2}
i) We calculate the implied volatility from Newton's method and PVF/disc with OLS as:
\begin{equation*}
\begin{array}{ccccc}
C_{Newton's Method} & C_{PVF disc} & Strike & P_{Newton's Method} & P_{PVF disc} \\
0.263136 & 0.257327 & 1175 & 0.254646 & 0.257170 \\ 
0.255054 & 0.249649 & 1200 & 0.246569 & 0.249214 \\ 
0.246974 & 0.241945 & 1225 & 0.238846 & 0.241629 \\ 
0.239118 & 0.234443 & 1250 & 0.231059 & 0.233998 \\ 
0.230649 & 0.226301 & 1275 & 0.223410 & 0.226529 \\ 
0.222679 & 0.218642 & 1300 & 0.215731 & 0.219059 \\ 
0.215221 & 0.211476 & 1325 & 0.208394 & 0.211967 \\ 
0.207559 & 0.204085 & 1350 & 0.200475 & 0.204336 \\ 
0.200157 & 0.196940 & 1375 & 0.192434 & 0.196645 \\ 
0.192422 & 0.189445 & 1400 & 0.184802 & 0.189441 \\ 
0.185341 & 0.182588 & 1425 & 0.177283 & 0.182454 \\ 
0.177794 & 0.175252 & 1450 & 0.170930 & 0.176761 \\ 
0.165550 & 0.163378 & 1500 & 0.154472 & 0.162372 \\ 
0.152350 & 0.150505 & 1550 & 0.138977 & 0.150782 \\ 
0.146540 & 0.144836 & 1575 & 0.129270 & 0.144739 \\ 
0.142885 & 0.141299 & 1600 & 0.118800 & 0.140195 \\ 
  \end{array}
\end{equation*}
ii) From the above table, we see that the implied volatilities for the calls (both methods) are within 0.01 of each other in most cases. This is also true for the implied volatilities of the puts. 
\vspace{5mm} \\
iii) From the table above, we see that directly using Newton's method with the traditional form of the Black Scholes equation gives a close answer to the OLS method with PVF and the discount factor. To verify that the implied volatility methods have been applied correctly, we expect the implied volatility for the call and put to be the same. Since we are using mid prices, we see that there is some noise in the above data due to that. As the implied volatilites calculated for the call or put at a certain strike are within 0.01 of each other, we can be assured that both methods are accurate.

\end{document}
